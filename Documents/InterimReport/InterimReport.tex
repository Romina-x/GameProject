\documentclass[]{final_report}
\usepackage{graphicx}
\usepackage{hyperref}
\usepackage{graphicx}
\usepackage{array}
\usepackage{xcolor} % Added package for text coloring
\usepackage{titlesec}

%%%%%%%%%%%%%%%%%%%%%%
%%% Input project details
\def\studentname{Romina Petrozzi}
\def\reportyear{2024}
\def\projecttitle{Building a Game}
\def\supervisorname{Julien Lange}
\def\degree{BSc (Hons) in Computer Science}
\def\fullOrHalfUnit{Full Unit} % indicate if you are doing the project as a Full Unit or Half Unit
\def\finalOrInterim{Interim Report} % indicate if this document is your Final Report or Interim Report

\begin{document}

\maketitle

%%%%%%%%%%%%%%%%%%%%%%
%%% Declaration

\chapter*{Declaration}

This report has been prepared on the basis of my own work. Where other published and unpublished source materials have been used, these have been acknowledged.

\vskip3em

Word Count: 

\vskip3em

Student Name: \studentname

\vskip3em

Date of Submission: 

\vskip3em

Signature:

\newpage

%%%%%%%%%%%%%%%%%%%%%%
%%% Table of Contents
\tableofcontents\pdfbookmark[0]{Table of Contents}{toc}\newpage

%%%%%%%%%%%%%%%%%%%%%%
%%% Your Abstract here

\begin{abstract}
- Problem \& objectives \newline
- Methods \newline
- Summarise what I achieved \newline

\end{abstract}
\newpage

%%%%%%%%%%%%%%%%%%%%%%
%%% Project Spec

\chapter{Project Specification}
\addcontentsline{toc}{chapter}{Project Specification}
Your project specification goes here.

%%%%%%%%%%%%%%%%%%%%%%
%%% Introduction
\chapter{Introduction}
- Background \& motivation \newline
- Game concept description \& mechanics \newline
- Objectives: Both for game and learning goals \newline

%%%%%%%%%%%%%%%%%%%%%%
%%% Literature Review
\chapter{Literature Review}
- Old abstract but more in depth
\paragraph{}
The video game industry has grown into one of the largest sectors in the entertainment industry, with video game sales being about five times higher than global music revenues, higher than consumer book sales, and similar to movie revenues (Marchand A, Hennig-Thurau T, 2013). This growth has been driven by many factors, including the growth of software, improvement of internet technology, and innovation (Zackariasson, P, Wilson T. 2012) . Video games have evolved significantly over the years, progressing from simple 2D graphics to immersive 3D experiences, and with the oncoming development of virtual reality (VR) and artificial intelligence (AI) technology, they are set to become even more immersive and engaging.
My motivation for this project stems from my interest in the VR industry. By working on this game with Unity 3D, I will gain experience using one of the most widely used engines for VR development and practice working with 3D environments. These skills will transition well into VR development, providing a good foundation for future projects in that field.
\paragraph{}
The game I am developing is a 3D platformer set in a fantasy forest, where the player controls an axe-wielding hero tasked with rescuing trapped animals guarded by goblin enemies. There will be a set number of levels, each with increasing difficulty that the player must traverse through. They will platform across obstacles by jumping over gaps, avoiding hazards, and fighting goblins. Points are awarded based on the number of enemies defeated in a level, and to clear each level the player must rescue all the animals being guarded. In addition, there will be extra collectables hidden throughout the levels which will grant the player bonus health and temporary power ups. This level design aims to cater to both ``segments of players labelled as `experiencers' versus `achievers''' (Yi Z. et al. 2022) by offering contained, linear levels which also encourage exploration. A highly successful example of this design is Super Mario 3D World, which my level designs will take inspiration from.
\paragraph{}
For the development of this game, I have chosen to use Unity game engine. `Game engines are platforms that make it easier to create computer games. They allow you to integrate and combine into single unit individual game elements such as animations, interaction with the user, or detection of collisions between objects' (Barczak M, Woźniak H, 2019). There are many game engines available for 3D development, with notable examples including Unity, Unreal Engine, and CryEngine. As Barczak and Wozniak (2019) explain, these engines provide reusable components, allowing developers to focus on gameplay and design rather than redeveloping fundamental systems. Unity stands out as the best choice for my project due to its powerful capabilities, user-friendly interface, and well documented resources. According to Christopoulou and Xinogalos (2017), Unreal Engine 4 and Unity are the two most developed engines, with Unreal Engine being more suited to experienced users providing remarkable graphics, while Unity is more suited to beginners with a large asset library and simpler user interface.
\paragraph{}
Optimisation techniques are crucial in game development, as performance lag can significantly impact the player's gameplay experience in many game genres. In particular, 3D platformers require precise movement and jumps for navigating levels and obstacles, and experiencing frame drops can make the gameplay frustrating for the player. Some of the main optimisation techniques used for games are level of detail (LOD) management, dynamic batching, occlusion culling (OC) and shader optimisation. The techniques I plan to use for my game are OC and LOD management. Occlusion culling ‘is the mechanism by which Unity avoids rendering computations for GameObjects that are fully hidden from view by other GameObjects’ (Singh, B. Sharma and A. Sharma, 2022). By not rendering objects outside of the player's view, performance is significantly improved as resources are not wasted on hidden elements. Unity has a built in function for this which I will make use of. LOD management adjusts the level of detail of an object depending on how far it is from the camera. In Unity, this is done using LOD levels, which specify the amount of detail of an object’s geometry to be rendered based on the object’s distance from the camera (Unity Technologies, 2024). By rendering fewer polygons for distant objects, the GPU workload is decreased, freeing up resources for closer objects.
\paragraph{}
Design patterns offer solutions to common problems faced by software developers. In game development, many common design issues have been solved through specific patterns that are widely used across many types of games. The use of these patterns helps maintain clean, readable code, making it easier to expand the game without disrupting existing functionality. Two common patterns that I will use in my game are State and Observer. The state pattern addresses player/object states for example, walking, idle and running and encapsulates each as an object. This pattern solves two problems: an object should change it’s behaviour when its internal state changes, and adding new states does not impact the behaviour of existing states (Unity Technologies, 2022). Within my game I will use this pattern for the player’s animation and movement transitions, as well as basic enemy AI states. The observer pattern is used when objects need to be notified without explicitly referencing them. This will be required when multiple different game objects need to be notified about events such as collecting a power up or defeating an enemy. The pattern involves observer and subject classes, where the observer contains a method that defines what action to take when notified by the subject, while the subject keeps a list of observers and notifies them when a specific event occurs (Nystrom R, 2014). This decouples the objects, allowing the subject to notify multiple observers without explicitly referencing them. 

%%%%%%%%%%%%%%%%%%%%%%
%%% Planning & Timeline
\chapter{Planning \& Timeline}
- Re write timeline from the project plan and indicate any changes from the original plan
The project timeline is divided into two terms. Term 1 focuses on developing and refining the core game mechanics, while Term 2 is dedicated to expanding features and optimising performance.

\subsection{Term 1}
\begin{tabular}{@{}p{2cm}@{}>{\raggedright\arraybackslash}p{0.5cm}@{}>{\raggedright\arraybackslash}p{0.5cm}@{}p{12cm}}
Week 1: & \textcolor{black}{|} & $\bullet$ & Learning and understanding the fundamentals of Unity game engine \\
Week 2: & \textcolor{black}{|} & $\bullet$ & Finalise the core game concept and gather all assets \\
Week 3-4: & \textcolor{black}{|} & $\bullet$ & Prototyping: Animations and movement for player and enemies\\
Week 5-6: & \textcolor{black}{|} & $\bullet$ &  Prototyping: Enemy interactions and health mechanics \\
& \textcolor{black}{|} & $\bullet$ & \textbf{Initial player and enemy movement prototype working by Week 7} \\
Week 7-8: & \textcolor{black}{|} & $\bullet$ & Design and User Interface: First level design and creating base user interface screens\\
Week 9: & \textcolor{black}{|} & $\bullet$ & Initial play-testing: Bug fixing and gathering feedback from users to refine core gameplay\\
& \textcolor{black}{|} & $\bullet$ & \textbf{Level one finished by Week 10} \\
Week 10-11: & \textcolor{black}{|} & $\bullet$ & Work on interim report and presentation \\
\end{tabular}

\subsection{Term 2}
\begin{tabular}{@{}p{2cm}@{}>{\raggedright\arraybackslash}p{0.5cm}@{}>{\raggedright\arraybackslash}p{0.5cm}@{}p{12cm}}
Week 1-3: & \textcolor{black}{|} & $\bullet$ & Feature Expansion: Multiple levels, score system, weapon upgrades\\
Week 4-5: & \textcolor{black}{|} & $\bullet$ & Audio and sound effects implementation \\
Week 6-7: & \textcolor{black}{|} & $\bullet$ &  Improvements to current graphics and game optimisation \\
Week 8: & \textcolor{black}{|} & $\bullet$ & Improving and expanding on current mechanics \\
Week 9-10: & \textcolor{black}{|} & $\bullet$ & Final play-testing : Bug fixing and gathering feedback from users to refine style and gameplay\\
& \textcolor{black}{|} & $\bullet$ & \textbf{Final game with expanded features finished by Week 10} \\
Week 10-11: & \textcolor{black}{|} & $\bullet$ & Work on final report \\
\end{tabular}
%%%%%%%%%%%%%%%%%%%%%%
%%% Development
\chapter{Development}
\section{Player Movement}
\paragraph{Hello}
- Player hierarchical state machine + UML \newline
- Strategy pattern for enemies + UML \newline
- Observer for combat system + UML \newline
\section{Enemy AI \& Combat}
- NavMesh agents for AI pathfinding for enemies \newline
- Cinemachine for smooth camera movement \newline
\section{Level Design}

%%%%%%%%%%%%%%%%%%%%%%
%%% Appendix#
\chapter{Appendix}
- Diary \newline
- Submission directory explanation \newline
- Demo video link\newline
%%%%%%%%%%%%%%%%%%%%%%
%%% References
\chapter{References}

Use one consistent system for citing works in the body of your report. Several such systems are in common use in textbooks and in conference and journal papers. Ensure that any works you cite are listed in the references section, and vice versa. 

%%%% ADD YOUR BIBLIOGRAPHY HERE
\newpage
\begin{thebibliography}{99}
\addcontentsline{toc}{chapter}{Bibliography}
\bibitem{COHEN:2013} Dave Cohen and Carlos Matos. \emph{Third Year Projects -- Rules and Guidelines}. Royal Holloway, University of London, 2013.
\end{thebibliography}
\label{endpage}



\end{document}

\end{article}
