\documentclass[12pt]{article}
\usepackage{graphicx}
\usepackage{fancyhdr}
\usepackage{amsmath}
\usepackage{amssymb}
\usepackage{hyperref}
\usepackage{enumitem}
\usepackage{array}
\usepackage{xcolor} % Added package for text coloring
\usepackage{titlesec}
\usepackage[a4paper, total={6in,8in}]{geometry}

\pagestyle{fancy}
\fancyhf{}
\fancyfoot[C]{\thepage}
\fancyhead[C]{\hrulefill} % Horizontal line in the header center
\fancyfoot[C]{\thepage}
\renewcommand{\headrulewidth}{0pt}

% Custom title format
\titleformat{\section}
  {\normalfont\fontsize{14}{17}\bfseries}{\thesection}{1em}{}

\titleformat{\subsection}
  {\normalfont\fontsize{12}{15}\bfseries}{\thesubsection}{1em}{}

\begin{document}

% Title page
\begin{titlepage}
    \centering
    \vspace*{5cm}
    \LARGE\textbf{Building a Game}\\
    \vspace{1cm}
    \hrulefill\\[0.5cm] % Horizontal line
    \large Project Plan\\
    \vspace{0.5cm}
    \large Romina Petrozzi\\
    \vspace{0.5cm}
    \large CS3810 - BSc Final Year Project\\
    \vspace{2cm}
    \large Supervised By: Julien Lange\\
    \vspace{0.5cm}
    \large Department of Computer Science\\
    \large Royal Holloway, University of London\\
    \vfill
\end{titlepage}



% Abstract
\section*{\centering Abstract}
\paragraph{}
The video game industry has grown into one of the largest sectors in the entertainment industry, with video game sales being about five times higher than global music revenues , higher than consumer book sales, and similar to movie revenues (Marchand A, Hennig-Thurau T, 2013). This growth has been driven by many factors, including the growth of software, improvement of internet technology, and innovation (Zackariasson, P, Wilson T. 2012) . Video games have evolved significantly over the years, progressing from simple 2D graphics to immersive 3D experiences, and with the oncoming development of virtual reality (VR) and artificial intelligence (AI) technology, they are set to become even more immersive and engaging.
\paragraph{}
The game I am developing is a 3D platformer set in forest ruins, where the player controls an axe-wielding hero tasked with rescuing trapped animals guarded by goblin enemies. There will be a set number of levels, each with increasing difficulty which the player must traverse through. They will platform across obstacles by jumping over gaps, avoiding hazards, and fighting enemy goblins. Points are awarded based on the number of enemies defeated in a level, and to clear each level the player must rescue all the animals being guarded. The level designs will take inspiration from Super Mario 3D World, with different obstacles and enemy situations to challenge the player. 
\paragraph{}
For the development of this game, I have chosen to use Unity game engine. ‘Game engines are platforms that make it easier to create computer games. They allow you to integrate and combine into single unit individual game elements such as animations, interaction with the user, or detection of collisions between objects’ (Barczak M, Woźniak H, 2019). There are many game engines available for 3D development, with notable examples including Unity, Unreal Engine, and CryEngine. As Barczak and Wozniak (2019) explain, these engines provide reusable components, allowing developers to focus on gameplay and design rather than redeveloping fundamental systems. Unity stands out as the best choice for my project due to its powerful capabilities, user-friendly interface, and well documented resources. According to Christopoulou and Xinogalos (2017), Unreal Engine 4 and Unity are the two most developed engines, with Unreal Engine being more suited to experienced users providing remarkable graphics, while Unity is more suited to beginners with a large asset library and simpler user interface.
\paragraph{}

\paragraph{}
My motivation for this project stems from my interest in the VR industry. By working on this game with Unity 3D, I will gain experience using one of the most widely used engines for VR development and practice working with 3D environments. These skills will transition well into VR development, providing a good foundation for future projects in that field.


% Timeline Section
\section{Timeline}
I decided to partition the project into two halves. The first half (Term 1) will primarily focus on enhancing understanding through literature reviews, acquiring and familiarising with the data. The latter half (Term 2) will concentrate on an in-depth analysis of the data, the application of data science techniques, and a review of the findings.

\subsection{Term 1}
\begin{tabular}{@{}p{2cm}@{}>{\raggedright\arraybackslash}p{0.5cm}@{}>{\raggedright\arraybackslash}p{0.5cm}@{}p{12cm}}
Week 1: & \textcolor{black}{|} & $\bullet$ & Learning and understanding the fundamentals of Unity game engine \\
Week 2: & \textcolor{black}{|} & $\bullet$ & Finalise the core game concept and gather all assets \\
Week 3/4: & \textcolor{black}{|} & $\bullet$ & Prototyping player and enemy movement \\
Week 5/6: & \textcolor{black}{|} & $\bullet$ &  Prototyping collision mechanics \\
Week 7/8: & \textcolor{black}{|} & $\bullet$ & Game architecture: Implementing design patterns \\
Week 9: & \textcolor{black}{|} & $\bullet$ & Initial playtesting and refinement of core gameplay mechanics : Bug fixing and gathering feedback from users\\
Week 10/11: & \textcolor{black}{|} & $\bullet$ & Work on interim report and presentation \\
\end{tabular}

\subsection{Term 2}
\begin{tabular}{@{}p{2cm}@{}>{\raggedright\arraybackslash}p{0.5cm}@{}>{\raggedright\arraybackslash}p{0.5cm}@{}p{12cm}}
Week 1-2: & \textcolor{black}{|} & $\bullet$ & stuff here \\
Week 3-4: & \textcolor{black}{|} & $\bullet$ & stuff here \\
Week 5-6: & \textcolor{black}{|} & $\bullet$ & stuff here \\
Week 7-8: & \textcolor{black}{|} & $\bullet$ & stuff here \\
Week 9-10: & \textcolor{black}{|} & $\bullet$ & stuff here \\
Week 11: & \textcolor{black}{|} & $\bullet$ & stuff here \\
\end{tabular}


% Risks and Mitigations
\section{Risks and Mitigations}
Any project you choose to undertake will inevitably encounter risks and require mitigations. I will start by discussing the general risks and their mitigations and then move on to those specific to my project.

\subsection{Hardware Failure}
Hardware failure

\subsection{Poor Estimation of Tasks}
Realistic time management is 

\subsection{Uneven Balance Between Report/Code}
The code is

\subsection{Machine Learning Risks}
With machine lea

\subsection{Overhead of Chemistry Knowledge}
As a compute

\subsection{Computational Efficiency}
It can be the ca

\subsection{Conscious Experiments}
Experim

% Acronyms
\section*{Acronyms}
\begin{itemize}
    \item CSP: Crystal Structure Prediction.
    \item DNN: Deep Neural Network.
    \item FUSE: Flexible Unit Structure Engine.
    \item ML: Machine Learning.
    \item SE: Software Engineering.
    \item TDD: Test-driven Development.
\end{itemize}

% Glossary
\section*{Glossary}
\begin{description}
    \item[GitHub] A cloud-hosted extension of a Git version control system with a web GUI.
    \item[Overleaf] An online LaTeX editor providing features such as history/version control and collaboration.
    \item[VESTA] A 3D visualization program for structural models, volumetric data such as electron/nuclear densities, and crystal morphologies.
\end{description}

% References
\section*{References}
\begin{enumerate}
	\item Hirsch, E., Graybiel, A.M. and Agid, Y.A. (1988) ‘Melanized dopaminergic neurons are differentially susceptible to degeneration in parkinson’s disease’, Nature, 334(6180), pp. 345–348.

	\item Chapuis, S. et al. (2004) ‘Impact of the motor complications of parkinson’s disease on the quality of life’, Movement Disorders, 20(2), pp. 224–230.


\end{enumerate}

\end{document}

