\documentclass[12pt]{article}
\usepackage{graphicx}
\usepackage{fancyhdr}
\usepackage{amsmath}
\usepackage{amssymb}
\usepackage{hyperref}
\usepackage{enumitem}
\usepackage{array}
\usepackage{xcolor} % Added package for text coloring
\usepackage{titlesec}


\pagestyle{fancy}
\fancyhf{}
\fancyfoot[C]{\thepage}
\fancyhead[C]{\hrulefill} % Horizontal line in the header center
\fancyfoot[C]{\thepage}
\renewcommand{\headrulewidth}{0pt}

% Custom title format
\titleformat{\section}
  {\normalfont\fontsize{14}{17}\bfseries}{\thesection}{1em}{}

\titleformat{\subsection}
  {\normalfont\fontsize{12}{15}\bfseries}{\thesubsection}{1em}{}

\begin{document}

% Title page
\begin{titlepage}
    \centering
    \vspace*{5cm}
    \LARGE\textbf{Building a Game}\\
    \vspace{1cm}
    \hrulefill\\[0.5cm] % Horizontal line
    \large Project Plan\\
    \vspace{0.5cm}
    \large Romina Petrozzi\\
    \vspace{0.5cm}
    \large CS3810 - BSc Final Year Project\\
    \vspace{2cm}
    \large Supervised By: Julien Lange\\
    \vspace{0.5cm}
    \large Department of Computer Science\\
    \large Royal Holloway, University of London\\
    \vfill
\end{titlepage}



% Abstract
\section*{\centering Abstract}
Parkinson's Disease (PD) is a progressive neurodegenerative disorder that significantly affects human life, impacting millions globally, particularly targeting ‘‘mobility, activities of daily living, stigma, and communicationwere" as found by  Chapuis et al. (2004). It is primarily characterized by the degeneration of dopaminergic neurons in the substantia nigra, a region of the brain responsible for movement control. Electroencephalograms (EEGs) are capable of recording the brain's electrical activity through non-invasive sensors placed on the scalp. The sensors will detect real-time changes in electrical signals near the surface of the brain. As stated by Etienne Hirsch (1988), there is a significant change between the tyrosine hydroxylase (TH) and neuromelanin (NM) neurons in brains of those with and without PD. Subjects who pertained PD were noted to have a decrease in TH by 77\% in the substantia nigra. 

%Mobility, Activities of Daily Living, Stigma, and Communicationwere the most strongly affected

% Timeline Section
\section{Timeline}
I decided to partition the project into two halves. The first half (Term 1) will primarily focus on enhancing understanding through literature reviews, acquiring and familiarising with the data. The latter half (Term 2) will concentrate on an in-depth analysis of the data, the application of data science techniques, and a review of the findings.

\subsection{Term 1}
\begin{tabular}{@{}p{2cm}@{}>{\raggedright\arraybackslash}p{0.5cm}@{}>{\raggedright\arraybackslash}p{0.5cm}@{}p{12cm}}
Week 1-2: & \textcolor{black}{|} & $\bullet$ & stuff here \\
Week 3-4: & \textcolor{black}{|} & $\bullet$ & stuff here \\
Week 5-6: & \textcolor{black}{|} & $\bullet$ & stuff here \\
Week 7-8: & \textcolor{black}{|} & $\bullet$ & stuff here \\
Week 9-10: & \textcolor{black}{|} & $\bullet$ & stuff here \\
Week 11: & \textcolor{black}{|} & $\bullet$ & stuff here \\
\end{tabular}

\subsection{Term 2}
\begin{tabular}{@{}p{2cm}@{}>{\raggedright\arraybackslash}p{0.5cm}@{}>{\raggedright\arraybackslash}p{0.5cm}@{}p{12cm}}
Week 1-2: & \textcolor{black}{|} & $\bullet$ & stuff here \\
Week 3-4: & \textcolor{black}{|} & $\bullet$ & stuff here \\
Week 5-6: & \textcolor{black}{|} & $\bullet$ & stuff here \\
Week 7-8: & \textcolor{black}{|} & $\bullet$ & stuff here \\
Week 9-10: & \textcolor{black}{|} & $\bullet$ & stuff here \\
Week 11: & \textcolor{black}{|} & $\bullet$ & stuff here \\
\end{tabular}


% Risks and Mitigations
\section{Risks and Mitigations}
Any project you choose to undertake will inevitably encounter risks and require mitigations. I will start by discussing the general risks and their mitigations and then move on to those specific to my project.

\subsection{Hardware Failure}
Hardware failure

\subsection{Poor Estimation of Tasks}
Realistic time management is 

\subsection{Uneven Balance Between Report/Code}
The code is

\subsection{Machine Learning Risks}
With machine lea

\subsection{Overhead of Chemistry Knowledge}
As a compute

\subsection{Computational Efficiency}
It can be the ca

\subsection{Conscious Experiments}
Experim

% Acronyms
\section*{Acronyms}
\begin{itemize}
    \item CSP: Crystal Structure Prediction.
    \item DNN: Deep Neural Network.
    \item FUSE: Flexible Unit Structure Engine.
    \item ML: Machine Learning.
    \item SE: Software Engineering.
    \item TDD: Test-driven Development.
\end{itemize}

% Glossary
\section*{Glossary}
\begin{description}
    \item[GitHub] A cloud-hosted extension of a Git version control system with a web GUI.
    \item[Overleaf] An online LaTeX editor providing features such as history/version control and collaboration.
    \item[VESTA] A 3D visualization program for structural models, volumetric data such as electron/nuclear densities, and crystal morphologies.
\end{description}

% References
\section*{References}
\begin{enumerate}
	\item Hirsch, E., Graybiel, A.M. and Agid, Y.A. (1988) ‘Melanized dopaminergic neurons are differentially susceptible to degeneration in parkinson’s disease’, Nature, 334(6180), pp. 345–348.

	\item Chapuis, S. et al. (2004) ‘Impact of the motor complications of parkinson’s disease on the quality of life’, Movement Disorders, 20(2), pp. 224–230.


\end{enumerate}

\end{document}

