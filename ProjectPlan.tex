\documentclass[12pt]{article}
\usepackage{graphicx}
\usepackage{fancyhdr}
\usepackage{amsmath}
\usepackage{amssymb}
\usepackage{hyperref}
\usepackage{enumitem}
\usepackage{array}
\usepackage{xcolor} % Added package for text coloring
\usepackage{titlesec}
\usepackage[a4paper, total={6in,8in}]{geometry}

\pagestyle{fancy}
\fancyhf{}
\fancyfoot[C]{\thepage}
\fancyhead[C]{\hrulefill} % Horizontal line in the header center
\fancyfoot[C]{\thepage}
\renewcommand{\headrulewidth}{0pt}

% Custom title format
\titleformat{\section}
  {\normalfont\fontsize{14}{17}\bfseries}{\thesection}{1em}{}

\titleformat{\subsection}
  {\normalfont\fontsize{12}{15}\bfseries}{\thesubsection}{1em}{}

\begin{document}

% Title page
\begin{titlepage}
    \centering
    \vspace*{5cm}
    \LARGE\textbf{Building a Game}\\
    \vspace{1cm}
    \hrulefill\\[0.5cm] % Horizontal line
    \large Project Plan\\
    \vspace{0.5cm}
    \large Romina Petrozzi\\
    \vspace{0.5cm}
    \large CS3810 - BSc Final Year Project\\
    \vspace{2cm}
    \large Supervised By: Julien Lange\\
    \vspace{0.5cm}
    \large Department of Computer Science\\
    \large Royal Holloway, University of London\\
    \vfill
\end{titlepage}



% Abstract
\section*{\centering Abstract}
\paragraph{}
The video game industry has grown into one of the largest sectors in the entertainment industry, with video game sales being about five times higher than global music revenues , higher than consumer book sales, and similar to movie revenues (Marchand A, Hennig-Thurau T, 2013). This growth has been driven by many factors, including the growth of software, improvement of internet technology, and innovation (Zackariasson, P, Wilson T. 2012) . Video games have evolved significantly over the years, progressing from simple 2D graphics to immersive 3D experiences, and with the oncoming development of virtual reality (VR) and artificial intelligence (AI) technology, they are set to become even more immersive and engaging.
\paragraph{}
The game I am developing is a 3D platformer set in forest ruins, where the player controls an axe-wielding hero tasked with rescuing trapped animals guarded by goblin enemies. There will be a set number of levels, each with increasing difficulty which the player must traverse through. They will platform across obstacles by jumping over gaps, avoiding hazards, and fighting enemy goblins. Points are awarded based on the number of enemies defeated in a level, and to clear each level the player must rescue all the animals being guarded. The level designs will take inspiration from Super Mario 3D World, with different obstacles and enemy situations to challenge the player. 
\paragraph{}
For the development of this game, I have chosen to use Unity game engine. ‘Game engines are platforms that make it easier to create computer games. They allow you to integrate and combine into single unit individual game elements such as animations, interaction with the user, or detection of collisions between objects’ (Barczak M, Woźniak H, 2019). There are many game engines available for 3D development, with notable examples including Unity, Unreal Engine, and CryEngine. As Barczak and Wozniak (2019) explain, these engines provide reusable components, allowing developers to focus on gameplay and design rather than redeveloping fundamental systems. Unity stands out as the best choice for my project due to its powerful capabilities, user-friendly interface, and well documented resources. According to Christopoulou and Xinogalos (2017), Unreal Engine 4 and Unity are the two most developed engines, with Unreal Engine being more suited to experienced users providing remarkable graphics, while Unity is more suited to beginners with a large asset library and simpler user interface.
\paragraph{}

\paragraph{}
My motivation for this project stems from my interest in the VR industry. By working on this game with Unity 3D, I will gain experience using one of the most widely used engines for VR development and practice working with 3D environments. These skills will transition well into VR development, providing a good foundation for future projects in that field.


% Timeline Section
\section{Timeline}
I decided to partition the project into two halves. The first half (Term 1) will primarily focus on enhancing understanding through literature reviews, acquiring and familiarising with the data. The latter half (Term 2) will concentrate on an in-depth analysis of the data, the application of data science techniques, and a review of the findings.

\subsection{Term 1}
\begin{tabular}{@{}p{2cm}@{}>{\raggedright\arraybackslash}p{0.5cm}@{}>{\raggedright\arraybackslash}p{0.5cm}@{}p{12cm}}
Week 1: & \textcolor{black}{|} & $\bullet$ & Learning and understanding the fundamentals of Unity game engine \\
Week 2: & \textcolor{black}{|} & $\bullet$ & Finalise the core game concept and gather all assets \\
Week 3-4: & \textcolor{black}{|} & $\bullet$ & Prototyping : Animations and movement for player and enemies\\
Week 5-6: & \textcolor{black}{|} & $\bullet$ &  Prototyping: Enemy interactions and health mechanics \\
Week 7-8: & \textcolor{black}{|} & $\bullet$ & Design and User Interface: First level design and creating base user interface screens\\
Week 9: & \textcolor{black}{|} & $\bullet$ & Initial playtesting : Bug fixing and gathering feedback from users to refine core gameplay\\
Week 10-11: & \textcolor{black}{|} & $\bullet$ & Work on interim report and presentation \\
\end{tabular}

\subsection{Term 2}
\begin{tabular}{@{}p{2cm}@{}>{\raggedright\arraybackslash}p{0.5cm}@{}>{\raggedright\arraybackslash}p{0.5cm}@{}p{12cm}}
Week 1-3: & \textcolor{black}{|} & $\bullet$ & Feature Expansion: Multiple levels, score system, weapon upgrades\\
Week 4-5: & \textcolor{black}{|} & $\bullet$ & Audio and sound effects implementation \\
Week 6-7: & \textcolor{black}{|} & $\bullet$ &  Improvements to current graphics and game optimisation \\
Week 8: & \textcolor{black}{|} & $\bullet$ & Improving and expanding on current mechanics \\
Week 9-10: & \textcolor{black}{|} & $\bullet$ & Final playtesting : Bug fixing and gathering feedback from users to refine style and gameplay\\
Week 10-11: & \textcolor{black}{|} & $\bullet$ & Work on final report \\
\end{tabular}


% Risks and Mitigations
\section{Risks and Mitigations}
Given the substantial scope of this game, there are several risks associated with its progress and completion. In this section, I'll discuss these risks and outline my strategies for mitigating them.

\subsection{Asset Acquisition}
Creating a 3D game requires numerous 3D models and assets. As I lack experience creating 3D models and animations, doing this myself would require significant extra time and present a very steep learning curve. This creates a significant risk as the quality and visuals of assets will impact the game’s graphic consistency and playability. To mitigate this risk, I will set dedicated time in the project for asset gathering. Prioritising the use of well-documented and optimised assets to ensure they integrate smoothly into the game.
\subsection{Learning New Tools}
As I have limited experience using Unity engine, learning this new software with its many capabilities could take a significant amount of time, which could impact the rate of game development. To mitigate this, I will make use of the vast amount of tutorials that exist, and dedicate some time early in the project to learn the fundamentals.
\subsection{Time Management}
There's a risk of falling behind schedule on both the project timeline and report writing. This is due to the fact that I have commitment to part time work and expect to spend most of my time working on the project code rather than the report. I will mitigate this by breaking down each timeline block into smaller, manageable tasks and prioritise the most important tasks. To make sure that I have sufficient time to complete the report, I will allocate time throughout the timeline rather than only at the end of each term to begin working on it gradually.
\subsection{Hardware Failure}
There is risk of the hardware I am using to work on the project to fail, such as a malfunctioning laptop or data loss, which could result in lost work.  To mitigate this, I will use Git version control system (VCS) to regularly back up my code and related files. This means that if my hardware fails I will be able to recover the project from my most recent commit.
\subsection{Hardware Constraints}
3D games generally require certain hardware specifications to run smoothly and avoid performance issues. As I will primarily be working on this project on a laptop, I may encounter performance issues if my device doesn't meet these specifications. To mitigate this, I will research optimisation strategies as challenges arise and consider using a more powerful machine when available. 
\subsection{Scope}
The ambitious scope of my project may prevent me from completing the game within the given timeframe. Given the complexity of developing a 3D game while simultaneously learning a new game engine at the same time could mean that I don’t achieve the desired features and visual quality. To mitigate this risk I will focus on establishing a minimum viable product (MVP) for the game by deciding on core, achievable aspects of the game. With this approach I can then build on these essential features as time allows.



% Acronyms
\section*{Acronyms}
\begin{itemize}
    \item CSP: Crystal Structure Prediction.
    \item DNN: Deep Neural Network.
    \item FUSE: Flexible Unit Structure Engine.
    \item ML: Machine Learning.
    \item SE: Software Engineering.
    \item TDD: Test-driven Development.
\end{itemize}

% Glossary
\section*{Glossary}
\begin{description}
    \item[GitHub] A cloud-hosted extension of a Git version control system with a web GUI.
    \item[Overleaf] An online LaTeX editor providing features such as history/version control and collaboration.
    \item[VESTA] A 3D visualization program for structural models, volumetric data such as electron/nuclear densities, and crystal morphologies.
\end{description}

% References
\section*{References}
\begin{enumerate}
	\item Hirsch, E., Graybiel, A.M. and Agid, Y.A. (1988) ‘Melanized dopaminergic neurons are differentially susceptible to degeneration in parkinson’s disease’, Nature, 334(6180), pp. 345–348.

	\item Chapuis, S. et al. (2004) ‘Impact of the motor complications of parkinson’s disease on the quality of life’, Movement Disorders, 20(2), pp. 224–230.


\end{enumerate}

\end{document}

